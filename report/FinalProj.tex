\documentclass{article}
\usepackage{verbatim}
\usepackage{amsmath}
\usepackage{graphicx}
\usepackage{listings}
\usepackage{color}
\begin{document}

%------------------------Title page---------------------------
\title{Final Project - Comprehensive Classifier Creation}
\author{Sam Neyhart\\ Jon Clark Freeman\\ Kevin Sayarath\\ ECE 471 - Professor Qi}
\date{April 23, 2017}
\maketitle
\newpage

%------------------------Abstract-----------------------------
\begin{abstract}
The objective of this final project was to integrate various classification
techniques to achieve the best performance on the chosen dataset. This project
was a group effort where we explored both supervised and unsupervised
classificaiton techniques including MPP (all 3 cases), kNN (with different k's),
BPNN (developed), decision tree, and k-means, winner-take-all, and kohonen maps.
We also explored two classifier fusion techniques including majority vote fusion
and naive Bayes.
\end{abstract}
\newpage


%------------------------Abstract-----------------------------
\begin{abstract}
This project serves as an introduction to unsupervised learning methods 
such as k-means, winner-take-all, Kohonen maps, and means-shift. In order
to gain a better understanding of how these algorithms work, the algorithms
are implemented in order to cluster the colors of the image. These images
are then compared in order to gain insight into how the various clustering
algorithms perform.
\end{abstract}
\newpage


%-----------------------Introduction---------------------------
\section*{Introduction}
\paragraph{}
This project performed for ECE 471 is an exercise all of the methods taught in this class learning.
By experimenting with various unsupervised learning methods, the colors
of the \textit{flowers.ppm} image are clustered into various numbers of clusters.
To form the clusters, first the k-means algorithm is used then the 
winner-take-all algorithm. The algorithms are implimented in python
using the Python Image Library (Pillow) package to handle image input and 
output.

\paragraph{} 
Overall the project served its purpose and was an opportunity to learn
much about the concepts of unsupervised learning. For this lab the 
PIL library was very helpful as it greatly simplified the input and output
of image data. The algorithm used for k-means was incredible time consuming.
\newpage


%-----------------------Technical Approach----------------------
\section*{Technical Approach}
\paragraph{K-means} 
The \textit{kmeans.py} file impliments the k-means algorithm using basic
python and the numpy library. It will perform k-means clustering on the 
data and then subsequently output the mean values for the requested number
of clusters. This implementation is incredible slow however so in order to
speed the process of obtaining the various versions of the \textit{flowers.ppm}
image required, the Project4.py file was created. 

\paragraph{Project4.py}
This project makes use of the file \textit{Project4.py} which contains 
instructions that perform k-means clustering on the image file using 
the PIL Image module which contains the function quantize. The quantize
function performs kmeans and then outputs a copy of the \textit{flowers.png} 
image consisting only of the reduced color set. The PIL image function is 
incredible fast compared to the k-means function I implimented. After verifying
that the two methods produce the same results for k=4, I chose to use the 
Project4.py file in order to perform the clustering.

\paragraph{Winner-take-all}
This project makes use of the file \textit{winner.py} which contains instructions
which implement winner-takes-all clustering. This algorithm is much faster than the
k-means algorithm. The winner-take-all clustering algorithm is initialized with random
cluster centers which can result in differing final values.

\paragraph{Comparison Metric}
In order to compare the methods, several comparison metrics were devised. 
The file size of the \textit{.png} images are compared, which provides some
insight into how the reduced color set allows for more compression. In addition
to these to metrics it seems that simple visual comparison is one of the most
powerful ways to understand the differences in the images.


%---------------------Experiments and Results-------------------------
\newpage
\section*{Experiments and Results}
%\subsection*{flowers.ppm} 
\newline
\centerline{\includegraphics[width = 4.5in]{../flowers}}
The original image \textit{flowers.ppm} is displayed above. All experiments
are performed on this image. The first expiriment is the kmeans clustering
of colors in \textit{flowers.png}. The resulting images can be seen on the following page. To generate these images k-means clustering was performed with [1, 2, 4, 8, 16
, 32, 64, 128, 256] clusters respectively.

\newpage
\centerline{
\includegraphics[width = 2in]{../kmeans/flowers_kmeans_1}
\includegraphics[width = 2in]{../kmeans/flowers_kmeans_2}
\includegraphics[width = 2in]{../kmeans/flowers_kmeans_4}}
\centerline{
\includegraphics[width = 2in]{../kmeans/flowers_kmeans_8}
\includegraphics[width = 2in]{../kmeans/flowers_kmeans_16}
\includegraphics[width = 2in]{../kmeans/flowers_kmeans_32}}
\centerline{
\includegraphics[width = 2in]{../kmeans/flowers_kmeans_64}
\includegraphics[width = 2in]{../kmeans/flowers_kmeans_128}
\includegraphics[width = 2in]{../kmeans/flowers_kmeans_256}}
The k-means images are seen above. The visual differences on these images are 
easy to see for the first couple images but by k=16 it becomes harder to find 
difference between the original image and the processed image. A list of the 
colors included in the each of these images is found in the apendix.
\newpage


%-----------------------------Discussion------------------------
\section*{Discussion}
\paragraph{}
This lab was very interesting to perform. Using python was a good
choice because it greatly simplified the k-means part of the lab.
The lab certainly increased my understanding of clustering algorithms 
and gave me greater insight to how and why they are used. The whole
concept of unsupervised learning is very interesting and it was fascinating
to see it in action here.
\newpage


%-----------------------------Reference------------------------
\section*{Discussion}
\paragraph{}
This lab was very interesting to perform. Using python was a good
choice because it greatly simplified the k-means part of the lab.
The lab certainly increased my understanding of clustering algorithms 
and gave me greater insight to how and why they are used. The whole
concept of unsupervised learning is very interesting and it was fascinating
to see it in action here.
\newpage


%----------------------------Appendix-----------------------------
\appendix
\section*{Appendix}
\subsection*{Project4.py}
\lstinputlisting[language=Python,numbers=left]{../Project4.py}
\newpage
\subsection*{kmeans.py}
\lstinputlisting[language=Python,numbers=left]{../kmeans.py}
\newpage
\subsection*{winner.py}
\lstinputlisting[language=Python,numbers=left]{../winner.py}
\newpage
\subsection*{kmeans-colors.txt}
\verbatiminput{../kmeans/colors.txt}
\newpage
\subsection*{wta-colors.txt}
\verbatiminput{../wta/colors.txt}
\end{document}
